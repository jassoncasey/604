\documentclass{article}
\usepackage{syntax}
\usepackage{bussproofs}
\usepackage{amsmath}
\usepackage{amssymb}
\usepackage{amsthm}
%\setlength{\grammarindent}{2.5cm}

\title{
Department of Computer Science \\Texas A\&M University}
\author{\LARGE Homework 3 Report\\ \\
\\Jasson Casey
\\Michael Lopez }

\begin{document}
\maketitle

\section{Typechecking}
SPL input files are read and typechecked. An AST is constructed from each file. The typechecker constructs a type context $\Gamma$, initializes a type variable generator and initializes the type substitution $\sigma$ to identity. At each node of the AST, the appropriate type rule is applied and $\Gamma$ and $\sigma$ are updated according to the rule. Unification is only executed when the APP rule is applied. Upon application of APP, the typechecker unifies the constraints gathered from each child AST and the result is composed with $\sigma$. This approach is inspired by the Damas-Milner system and algorithms found in the Mini-ML paper.


\section{Typing Relations}

The rules are inspired by the modified Damas-Milner system and includes support for natural numbers, primitives and the list constructors.

\begin{center}
\begin{prooftree}
\RightLabel{{T-TAUT}}
\AxiomC{$x:\tau \in \Gamma$}
\UnaryInfC{$\Gamma \vdash x : \tau$}
\end{prooftree}

\begin{prooftree}
\RightLabel{{T-NAT}}
\AxiomC{$n \in N$}
\UnaryInfC{$\Gamma \vdash n : \mbox{Nat}$}
\end{prooftree}

\begin{prooftree}
\RightLabel{{T-ARITH}}
\AxiomC{$s \in \{ +,-,*,/ \}$}
\UnaryInfC{$\Gamma \vdash s : \mbox{Nat} \rightarrow \mbox{Nat} \rightarrow \mbox{Nat}$}
\end{prooftree}

\begin{prooftree}
\RightLabel{{T-CONS}}
\AxiomC{}
\UnaryInfC{$\Gamma \vdash \mbox{cons} : \tau \rightarrow [\tau] \rightarrow [\tau]$}
\end{prooftree}

\begin{prooftree}
\RightLabel{{T-CONS}}
\AxiomC{}
\UnaryInfC{$\Gamma \vdash \mbox{nil} : [\tau]$}
\end{prooftree}

\begin{prooftree}
\RightLabel{{T-APP}}
\AxiomC{
  $\Gamma \vdash e : \tau' \rightarrow \tau$
  \hspace{6mm}
  $\Gamma \vdash e' : \tau'$}
\UnaryInfC{$\Gamma \vdash e \hspace{1mm} e' : \tau$}
\end{prooftree}

\begin{prooftree}
\RightLabel{{T-ABS}}
\AxiomC{
  $\Gamma_x \cup \{x:\tau'\}\vdash e : \tau$}
\UnaryInfC{$\Gamma \vdash \lambda x. e : \tau' \rightarrow \tau$}
\end{prooftree}

\begin{prooftree}
\RightLabel{{T-LET}}
\AxiomC{
  $\Gamma \vdash e' : \tau'$
  \hspace{6mm}
  $\Gamma_x \cup \{ x : \sigma \} \vdash e : \tau$}
\UnaryInfC{$\Gamma \vdash \mbox{ let } x = e' \mbox{ in } e : \tau$}
\end{prooftree}
\end{center}

\section{Evaluation}

Before the AST is evaluated, a core set of functionals (cons, nil, etc.) are
evaluated and bound to the global scope. The AST is then evaluated using the
call-by-value strategy. After the right side is a value (irreducible),
a beta transformation is applied and the evaluator descends into the expression
on the left side of an application. When descent is no longer possible, the
evaluator returns to the scope of the root of the resulting expression.

\subsection{Operational Semantics}

Some computation steps are carried out as $\delta$ functions. $\Delta$
identifies the set of tuples representing the mapping from domain
to range of the $\delta$ function. Currently, the $\delta$ function is
defined for $( +, -, x, / )$ over $\mathbb{N}$.

\begin{prooftree}
\RightLabel{\textbf{$E_{seq1}$}}
\AxiomC{}
\UnaryInfC{$v$; $t\rightarrow t$}
\end{prooftree}
 
\begin{prooftree}
\RightLabel{\textbf{$E_{app1}$}}
\AxiomC{}
\UnaryInfC{$( \lambda x. t_1 )$ $v_2 \rightarrow$[$ x \mapsto v_2$] $t_1$}
\end{prooftree}

\begin{prooftree}
\RightLabel{\textbf{$E_{app2}$}}
\AxiomC{$t_2 \rightarrow t_2'$}
\UnaryInfC{$t_1$ $t_2 \rightarrow t_1$ $t_2'$}
\end{prooftree}
 
\begin{prooftree}
\RightLabel{\textbf{$E_{app3}$}}
\AxiomC{$t_1 \rightarrow t_1'$}
\UnaryInfC{$t_1$ $v_2 \rightarrow t_1'$ $v_2$}
\end{prooftree}

\begin{prooftree}
\RightLabel{\textbf{$E_{seq2}$}}
\AxiomC{$t_1 \rightarrow t_1'$}
\UnaryInfC{$t_1$ ; $t_2 \rightarrow t_1'$ ; $t_2$}
\end{prooftree}

\begin{prooftree}
\RightLabel{\textbf{$E_{\delta1}$}}
\AxiomC{$t_1 \rightarrow t_1'$}
\UnaryInfC{$\delta_{op}$ $t_1$ $t_2 \rightarrow \delta_{op}$ $t_1'$ $t_2$}
\end{prooftree}

\begin{prooftree}
\RightLabel{\textbf{$E_{\delta2}$}}
\AxiomC{$t_2 \rightarrow t_2'$}
\UnaryInfC{$\delta_{op}$ $v_1$ $t_2 \rightarrow \delta_{op}$ $v_1$ $t_2'$}
\end{prooftree}

\begin{prooftree}
\RightLabel{\textbf{$E_{\delta3}$}}
\AxiomC{$((\delta_{op}$, $v_1$, $v_2), v_3) \in \Delta$}
\UnaryInfC{$\delta_{op}$ $v_1$ $v_2 \rightarrow v_3$}
\end{prooftree}

\section{Testing}

The type inference and argument deduction was tested against a large set of 
test inputs designed using the type construction rules from the Mini-ML paper.
Negative and positive test cases were created directly from the syntax directed
typing rules outlined.

\section{Files}
\begin{tabular}{|p{20mm}|p{93mm}|}
  \hline
{\bf Files}&{\bf Description}\\
  \hline
spl.hs & Main file that initializes the SPL typechecker.\\
  \hline
Lexing.hs & Module that implements lexing and defines tokens for SPL.\\
  \hline
Parsing.hs & Module that implements the parser for SPL.\\
  \hline
Ast.hs & Module that defines the AST data structure for SPL.\\
  \hline
Eval.hs & Module containing evaluation facilities for the SPL.\\
  \hline
Typing.hs & Module that implements type inference and type error reporting.\\
  \hline
ProofTree.hs & Module that implements type inference and returns the proof tree.\\
  \hline
\end{tabular}

\end{document}
