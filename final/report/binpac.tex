\section{Binpac}

Binpac~\cite{binpac} can be viewed as an extension of
PacketTypes~\cite{packet_types}, the authors' primary view is that
protocol message formats are merely a grammar problem, and seek
to generalize a format description language into a `data-model' with
constraints. The problem is described in the same terms as with PacketTypes:
tedious, error-prone, hard-to-read, etc. The generalization includes support for
ASCII based protocols which was not present with previous 
work~\cite{packet_types}. The target application is off-line packet capture
analysis.

\subsection{Contributions}

The authors' identity several avenues of attack. First, they want to provide for
parser generation for network protocol message formats that covers both binary
and ASCII based protocol formats. Second, they want the ability to have
resumable state parsers (the example given involves TCP processing where the
application message spans more than one TCP packet). Third, they identify the
need to perform optimizations on code generation to synthesis efficient code.

Binpac defines the elementary types empty and int/uint{8/16/32}. A string type
can be defined with a string literal, a regular expression, or bytestring. 
The composite types are: array, record, and case (variant). The case type is
uniquely indexed and does not require look-ahead (unlike with packet types).
Binpac introduces support for state management in order to handle inter-packet
decodes and manage multiple unique flows. Finally, `refinement' is supported
which allows field subclasses of records and variants.

Performance was evaluated using a generated and hand coded parser for HTTP and
DNS. These parsers were then run over traffic collected on a local network. The
authors' reported throughput, performance, and lines of code (LOC). Binpac always
produced smaller code footprints, but were almost equivalent in throughput and
performance.

\subsection{Critique}

It is strange that the author's treat that is `yacc' for network protocol
implementation. This almost implies that their view of this problem is purely
with producing a parse from a grammar. This philosophy would omit a large
subset of bugs that arise from the improper use of message structures.

All array and variant access is checked at runtime by generated code. This could
lead to poor performance. If field access can be determined to be safe at
compile time then there is no need for run-time checks. Run-time checks are only
necessary for single occurrences of unchecked code at compile time. Complete 
run-time checking is necessary when type information is lost by having a 2-tier
type system (binpac, c++).

By treating protocol messages as second class citizens binpac misses the
opportunity to reason about the use of messages at compile time. This seems to
be one of the most compelling reasons for capturing protocol message structures
in a language.
