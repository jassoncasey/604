\section{Intuition}

The following example uses a simple message format similar to what is depicted
in ~\ref{fig:fig1}. This example captures a rich set of the dependencies present
in network protocols. A message starts with a fixed width field, called opcode,
that represents an unsigned integer that is 16 bits wide. A second field, called
payload, is optionally present and depends on the value of opcode. If the opcode
contains the value 2, then the payload is present and is an unsigned integer 16
bits wide. If the opcode is not 2 then payload is a 0 bit width type, which 
implies its absence. Just as with handling network packets, accessing the 
payload is not safe without first verifying its presence.

\begin{lstlisting}[language=Haskell]
-- Dependent record packing precision types
pdu Pkt {
   opcode  = Uint 16
   payload = Uint (opcode==2 ? 16 : 0)
}

-- Unsafe dependent access
test :: Pkt -> Bool
test pkt = pkt.payload == 0

-- Safe dependent access
test :: Pkt -> Bool
test pkt = if pkt.opcode == 2 
            then pkt.payload == 0 
            else False
\end{lstlisting}

\subsection{Pdu}



\subsection{Uint}

\subsection{Dependency}

