\section{Intuition}

The following example uses a simple message format similar to what is depicted
in ~\ref{fig:fig1}. This example captures a rich set of the dependencies present
in network protocols. A message starts with a fixed width field, called opcode,
that represents an unsigned integer that is 16 bits wide. A second field, called
payload, is optionally present and depends on the value of opcode. If the opcode
contains the value 2, then the payload is present and is an unsigned integer 16
bits wide. If the opcode is not 2 then payload is a 0 bit width type, which 
implies its absence. Just as with handling network packets, accessing the 
payload is not safe without first verifying its presence.

\begin{lstlisting}[language=Haskell, caption={Message Format Usage Example}]
-- Dependent record packing precision types
pdu Pkt {
   opcode  = Uint 16
   payload = Uint (opcode==2 ? 16 : 0)
}

-- Unsafe dependent access
test :: Pkt -> Bool
test pkt = pkt.payload == 0

-- Safe dependent access
test :: Pkt -> Bool
test pkt = if pkt.opcode == 2 
            then pkt.payload == 0 
            else False
\end{lstlisting}

\subsection{Pdu}

The pdu type is similar to a record or structure from other languages. It
consists of a series of label to type bindings. This type is meant to allow
definition of precise width composite types for overlaying with a raw memory
buffer received from some form of IO. This implies that implicit padding of
elements within the pdu for byte, word, double word, and quad word alignment
are strictly prohibited. The first element begins at bit offset 0 from the base
of the memory butter, and the ith element begins at the offset defined by the
bit width of its preceding elements.

\subsection{Uint}

The Uint type represents an unsigned integer of precise width. This type is 
parametric over a term that defines its width. In the case of opcode from the
previous example the term is a constant and is clearly determined at compile
time. However, payload's Uint term is a conditional expression that changes
the size of the payload based on the value contained within opcode. The precise
size of payload cannot be determined at compile time, but only by the reception
of a packet.

\subsection{Dependency}

There are two forms of dependency expressed in this example. First, Uint is a 
dependent type, its width may depend on any valid term. Second, pdu is a
dependent record type, meaning any label's type may depend on any of the
preceding labels. 

The compiler will try and assert that access to any type with a dependency at
compile time is safe. This is rather difficult when run-time dependencies are
involved. However, parameters of a function, data constructor capture from case,
and predicates from conditional expressions all provide some compile time
information regarding data values. For instance in the example from above the 
then block of the condition expression is predicated on $pkt.opcode == 2$, which
happens to be the same dependency that must be satisfied for safe access to 
$pkt.payload$. Therefore, regardless of any input packet this access is always
safe, and this may be determined at compile time.

The unsafe access is illustrated in the same example. The current implementation
of Steve will produce a compile time failure for this access. However, this may
not be the most useful of actions. Permuting unsafe access to guaranteed safe
access at compile time is currently being explored.
