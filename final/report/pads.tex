\section{PADS}

This is a position paper explaining the existing work on 
the PADS project. PADS is a declarative overlay implemented for: C, ML, and
Haskell as a work in progress. The goal for PADS is to capture the richness
of ad-hoc data formats: ASCII, binary, Cobol, and `mixed data formats.' The 
targeted user of the PADS system are system administrators who deal with large
variations and constant change of ad-hoc data formats. The authors give many
examples detailing various XML and log file processing requirements of network
service providers.

\subsection{Contributions}

A PADS description compiles to produce three data structures: a memory layout
representation, a meta-data description, and error collection. Additionally, 
a parser and pretty print function are generated by the compile. It appears that
the process will parse an entire input and collect all errors before returning
control. It appears that serializing back to a file is currently being
developed.

The generated parser is a recursive-decent parser. There are known limitations 
with this style parser with respect to different grammars. The authors' comment
that `always-efficient' parsing is an active research question.

The Data Description Calculus (DDC)~\cite{next_700_ddl} is referenced as the 
formal underpinnings of the PADS system.

The PADS systems ships with a few tools that work `for-free' with any PADS
description. Accumulator is a statistics generator, which can run against
a PADS description to produce min/max/avg. values across numeric values,
lengths of strings, summaries of structures, etc.  occurring within the
description. XML converter can render an XML file from a data object and its
associated description. Relational converter performs a similar task, but 
for relational databases. XQuery allows for running queries against large
collections of de-serialized data objects, and finally they have a data 
synchronization tool integration ability for `Harmony'.

The authors' reference another paper~\cite{dirt_to_shovels} that shows the
ability to infer PADS descriptions in a user directed fashion, based on
acquiring large sets of data objects. The authors' are currently working on
improved automated inference~\cite{inf_syslog}, and human directed inference
called ANNE~\cite{anne}. ANNE is based on Relevance Logic, instead of type
theory, to provide better expressiveness proofs.

\subsection{Critique}

PADS seems to be well suited for off-line processing of data. The generated 
parse function attempts to de-serialize the entire data object. This is not
appropriate for on-line processing where low latency is a requirement, and
processing the entire data object is not required. This approach would not
be suitable for high performance network applications. 

The automated tools described above are compiler generated, and not the result
of generic programming. This limits the audience of people wanting to extend
generic tools. The authors' note this deficiency and intend to move to Haskell
as a base language to natively support this facility.

Performance for a PADS/C output is compared to hand coded perl routine, and as 
you might expect results in a faster execution. The argument used by the authors
is that its from their user community; however, its not a valid comparison model.

User is exposed to understanding generated libraries, problem with any overlay
system.

\subsection{Future Work}

The authors' identify format inference as an active research area where they
will focus a large portion of the time. Additionally, they are concerned with
Haskell integration in order to benefit from active work in the language
community.
