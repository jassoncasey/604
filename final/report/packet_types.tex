\section{Packet Types}

PacketTypes~\cite{packet_types} is a first work towards developing a
language to handle the specific constraints of network protocol
message formats. The authors' motivate the work with the usual suspects
of networkk protocol software development: tedious, error-prone, ever-changing,
etc. Users can specific binary formatted network protocols in the PacketTypes
overlay language. This language is then compiled down to a set of C
datastructures and functions. Type checking is performed at the overlay level,
while runtime checks are provided through their generated C functions. 

\subsection{Contributions}

The primary contribution of this paper is a network protocol message format
language. This language includes the following primitives: bit, array, record,
constraints, refinement, and a variant type. User defined types are supported 
by assigned a closed type constructor to an identifier. Arrays by have either
a constant integral size, or are unbounded. Unbounded arrays will match all
binary data unless there is a size constraint present. Records have the usual
structure where a set of unique labels may be assigned types. Constraints
apply to records and may constrain values and types. Constraints are defined
as assignments or predicates of expressions values or type parameters.
Expression may contain any previously encountered identifier. Refinement is a
form of psuedo-subclassing where selected values and types of the super type may
be overridden. The variant provides a `alternation' of types that may be
by first match. Types may have attributes that can be used in constraint
expressions. All types seem to have the attributes: value, numbits, and
numbytes. Arrays have numelements, and variants have an attribute called alt
that refers tag of a constructed variant. Expressions support the following
operators: $=$, $\neq$, $<$, $>$, $\leq$, $\geq$, $||$, $\&\&$, $\times$,
$/$, $+$, and $-$.

Literals may be used to identify a unique type width and value. These types
only appear inside of `alternation' constructions.

\subsection{Critique}

Constraints are expressed outside of the context of the constrained type,
this can be confusing, its not clear why these constraints are not expressed
directly in the constrained object. This would be possible in their scheme
by not using the greedy array but [ constraint predicate ] instead. This is
a minor critique but seems to increase readability of their language.

It is not clear how types are given default initializers or even if this is
a supported feature.

The `alternation' type matches the first of its potential terms in a choice,
its not obvious that this match may be directed. Again, it is not clear why
matching is not obvious within the type definition as this would improve
readability of the description. Also, it is possible that there are not enough
constraints in the inner structure and values of types to always allow for
a disjoint matching condition. In this scenario order of the type definition
can change the resulting match, which would not be easily understood behavior.

Ultimately, this approach depends on implicit matching as opposed to a guided
protocol decode. While the goal is to simply the programmer's role of writing
format handling code, the tool should still resemble the high level
format specification. When a matching approach is taken this objective does not
seem possible. Also, it is not clear that performance can be maintained under
matching occurrence within an `alternation' versus being directed to a specific
choice. Additionally, refinement matches are all searched for the longest match,
or `strongest' condition. It is hard to imagine how this is efficient in a 
system with many active protocol definitions.

There are no apparent methods being used during type checking or code generation
of the packet types compiler. This high level process described does not seem
to be based on any principled mechanisms.

The paper refers to refinement selection as `demultiplexiing' and indicates that
the compiler cannot handle demultiplexing across ranges. Additionally, the
compiler assumes n-byte alignment on n byte boundaries, which is not the case
many active protocols.

The algorithms at use in this paper are rather obtuse and not clearly defined. 
The only thing with a formal definition is the grammar of the language itself.

* not sure if the array allows non-const expression as size
* overlay language, benefit of type checking is lost by types
  not carrying over to the execution language
* one shot parse
* authors keep switching between syntax and semantic arguments
  without treating them seperately
* are alternations/variants truely disjoint?
