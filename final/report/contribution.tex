\newcommand{\pn}[1]{ \textnormal{#1} }
\newcommand{\oo}{\; \mid \;}
\newcommand{\nb}{~~~|~}
\newcommand{\sk}{\dots }
\newcommand{\ww}{\;}
\newcommand{\nn}{\perp}
\newcommand{\sm}[1]{\textnormal{#1}}
\newcommand{\sd}[1]{\textnormal{\it #1}}

% Fix spacing on grammar

\section{Language Specification}
In this section we present the language specification for the Steve Programming
Language. The language uses a design where only four types of declarations can
appear top level.

\subsection{Grammar}
In this section we present the grammar that is currently in the Steve Language.
The grammar reflects the top-level design

\subsubsection{Top-level Declarations}

\begin{align*}
  \pn{prgm} ::&=~ \pn{datatype} \\
         &~~|~ \pn{function} \\
         &~~|~ \pn{empty} \\
  \pn{function} ::=&~ \pn{annotation} ~ \pn{func-defn} \\
  \pn{datatype} ::=&~ \pn{pdu} \\
             &~~|~ \pn{adt} \\
  \pn{enum} ::=& \pn{\emph{not yet decided}}
\end{align*}

\subsubsection{Functions and Expressions}

\begin{align*}
  \pn{annotation} ::=&~ \pn{identifier :: type-expr ;}\\
  \pn{func-defn} ::=&~ 
    \pn{identifier id-seq = func-body ;}\\
  \pn{func-body} ::=&~ \pn{case-expr} \\
  \pn{case-expr} ::=&~  \pn{\emph{case} identifier \emph{of} \emph{\{} case-clauses \emph{\}}}\\
  \pn{case-clause-seq} ::=&~  \pn{case-clause-seq \emph{;} case-clause}\\
    &~~|~ \pn{case-clause}\\
  \pn{case-clause} ::=&~ 
    \pn{typename parameter-seq $\rightarrow$ case-expr}\\
  \pn{if-expr} ::=&~ 
    \pn{\emph{if} conditional-expr \emph{then} expr \emph{else} expr}\\
  \pn{conditional-expr} ::=&~ 
    \pn{conditional-expr \emph{and} neg-expr}\\
    &~~|~ \pn{conditional-expr \emph{or}  neg-expr}\\
  \pn{neg-expr} ::=&~ 
    \pn{\emph{not} cmpr-expr}\\
    &~~|~ \pn{cmpr-expr}\\
  \pn{cmpr-expr} ::=&~ 
    \pn{cmpr-expr $<$ plus-expr}\\
    &~~|~ \pn{cmpr-expr $>$ plus-expr}\\
    &~~|~ \pn{cmpr-expr $<=$ plus-expr}\\
    &~~|~ \pn{cmpr-expr $<=$ plus-expr}\\
  \pn{plus-expr} ::=&~ 
    \pn{plus-expr \emph{+} times-expr}\\
    &~~|~ \pn{plus-expr \emph{-} times-expr}\\
  \pn{negative-expr} ::=&~ 
    \pn{\emph{-} times-expr}\\
    &~~|~ \pn{times-expr}\\
  \pn{times-expr} ::=&~ 
    \pn{times-expr \emph{*} applicative-expr}\\
    &~~|~ \pn{times-expr \emph{/} applicative-expr}\\
    &~~|~ \pn{times-expr \emph{\%} applicative-expr}\\
  \pn{applicative-expr} ::=&~ 
    \pn{adtConstructor}\\
    &~~|~ \pn{applicative-expr primary}\\
    &~~|~ \pn{primary}\\
  \pn{primary} ::=&~ 
    \pn{identifier}\\
    &~~|~ \pn{literal}\\
    &~~|~ \pn{pduConstructor}\\
    &~~|~ \pn{(expr)}
\end{align*}

\subsubsection{Type Expressions}

\begin{align*}
  \pn{type-expr} ::=&~
    \pn{type-expr $\rightarrow$ type-term}\\
    &~~|~ \pn{type-term}\\
  \pn{type-term} ::=&~
    \pn{(type-expr)}\\
    &~~|~ \pn{[type-expr]}\\
    &~~|~ \pn{complex-type}\\
    &~~|~ \pn{\emph{Nat}}\\
    &~~|~ \pn{\emph{Bool}}\\
    &~~|~ \pn{\emph{Char}}\\
    &~~|~ \pn{identifier}\\
  \pn{complex-type} ::=&~
    \pn{typename expr-seq} \\
    &~~|~ \pn{\emph{Uint} expr expr} \\
    &~~|~ \pn{\emph{Array} type-term expr}
\end{align*}

\subsubsection{Data Declarations}

\begin{align*}
  \pn{pdu-declaration} ::=&~
    \pn{\emph{pdu} typename \emph{=} \emph{\{} pdu-field-seq \emph{\}}}\\
  \pn{pdu-field-seq} ::=&~
    \pn{pdu-field-seq , pdu-decl-field}\\
    &~~|~ \pn{pdu-decl-field}\\
  \pn{pdu-decl-field} ::=&~
    \pn{identifier \emph{:} type-expression}\\
  \pn{adt-declaration} ::=&~
    \pn{"data" identifier \emph{=} adt-ctor-sequence \emph{;}}\\
  \pn{adt-ctor-sequence} ::=&~
    \pn{adt-ctor-sequence \emph{|} adt-ctor}\\
    &~~|~ \pn{adt-ctor}\\
  \pn{adt-ctor} ::=&~
    \pn{typename type-sequence}\\
  \pn{type-sequence} ::=&~
    \pn{type-sequence type-term}\\
    &~~|~ \pn{type-term}\\
    &~~|~ \pn{empty}
\end{align*}

\subsection{Formal Description}
The internal language defines a term as:
  \begin{flushleft}
  \begin{align*}
  t &::= t~t ~|~ \lambda x:\tau.t ~|~ x ~|~ c ~|~
    \mbox{if } t \mbox{ then } t \mbox{ else } t \\
     &~~|~
    \mbox{let } x:\tau = t \mbox{ in } t ~|~ 
    K~\overline{t} ~|~
    \mbox{case } t \mbox{ of } \{
      \overline{K~\overline{t} \rightarrow t} \}\\
    &~~|~
    \mbox{uint } t~t ~|~ \mbox{ array } \tau~t ~|~ t[t] ~|~ t.\ell ~|~
    id \{ \overline{\ell = t} \} \\
    &~~|~
    \mbox{data } id = \overline{K \overline{\tau}} ~|~
    \mbox{pdu } id = \{ \overline{\ell = \tau} \} \\
    &~~|~ \mbox{enum } id = \{\overline{\tau \rightarrow  \mbox{Bool}, \tau}\}
  \end{align*}
\end{flushleft}
In addition to types, we also reason about kinds.
\begin{flushleft}
   \begin{align*}
      K &::= \star ~|~ \Pi x:\tau.K
   \end{align*}

  \begin{align*}
    \tau &::= \mbox{ Ascii } | \mbox{ Nat } | \mbox{ Bool } \\
         &~~|~ \Pi x:\tau.\tau ~|~ [\tau] \\
         &~~| \mbox{ Uint } t~t ~|~ \mbox{Array } \tau~t \\
         &~~| \mbox{ Pdu } \{\overline{\ell = \tau}\} ~|~ 
         \mbox{Enum } \{\overline{\tau \rightarrow  \mbox{Bool}, \tau}\} 
  \end{align*}
\end{flushleft}

The Steve Programming Language is a statically typed language that supports
dependent types and records. Type binding and kind binding information are
stored in $\Gamma$ and user defined type information is stored in $\Sigma$.
Steve also has a fact environment $\Delta$ responsible for collecting static
properties of the program.

Enums are still a work in progress and ADTs are not yet implemented.

\subsubsection{Top-level Typing}

  \begin{mathpar}
    \inferrule[TopLevel Record]{
      \Sigma ~|~ \Gamma \vdash \mbox{ pdu } id \{ \overline{\ell_i = \tau_i} \} : \rho \\
      \Sigma, \langle id : \rho \rangle ~|~ \Gamma, \langle id::\star \rangle  \vdash prgm
    } {
      \Sigma ~|~ \Gamma \vdash \mbox{ pdu } id \{ \overline{\ell_i = \tau_i } \} , prgm  
    }\\
   \inferrule[Type TopLevel Let]{
      \Sigma | \Gamma \vdash \tau ::\star \\
      \Delta | \Sigma | \Gamma \vdash t : \tau \\
      \Delta | \Sigma | \Gamma, \langle f:\tau \rangle \vdash prgm
   } {
      \Delta | \Sigma | \Gamma \vdash \mbox{let }f = t:\tau, prgm 
   }
\end{mathpar}

\subsubsection{Fact Rules}

  \begin{mathpar}
    \inferrule[Conditional]{
      \Delta | \Sigma | \Gamma | \theta \vdash t_1 : \mbox{Bool}
      \rightsquigarrow \theta' \\
      \Delta, \langle t_1 \rangle | \Sigma | \Gamma | \theta' \vdash
      t_2 : \tau \rightsquigarrow \theta'' \\
      \Delta, \langle t_1 \rangle | \Sigma | \Gamma | \theta'' \vdash
      t_3 : \tau \rightsquigarrow \theta''' \\
    } {
      \Delta | \Sigma | \Gamma | \theta \vdash \mbox{if } t_1
      \mbox{ else } t_2 \mbox{ then } t_3 : \tau
      \rightsquigarrow \theta'''
    }
  \end{mathpar}

  \begin{mathpar}
    \inferrule[Term App]{
      \Delta | \Sigma | \Gamma | \theta \vdash t_1 : \Pi x:\tau_1.\tau_3 
      \rightsquigarrow \theta'\\
      \Delta | \Sigma | \Gamma | \theta' \vdash t_2:\tau_2 
      \rightsquigarrow \theta''\\
      bw \leftarrow BoundVar(\tau_1)~
      bv' \leftarrow Fresh(bv) \\
      \Delta | \Sigma | \Gamma | \theta'' \vdash [bv'/bv]Body(\tau_1) \equiv \tau_3 
      \rightsquigarrow \theta'''
    } {
      \Delta | \Sigma | \Gamma | \theta \vdash t_1~t_2:[t_2/x][bv'/bv]\tau 
      \rightsquigarrow \theta'''
    }
  \end{mathpar}