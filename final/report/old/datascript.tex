\section{DataScript}

DataScript~\cite{datascript} is an overlay language designed to provide 
declarative ad-hoc data format specifications. This work aimed to advance the 
work of PacketTypes for more general data formats as well as generate
traversal functions for verified data. This language targeted the Java
runtime environment.

\section{Contributions}

The paper presents the following categories of language features:
primitive types, set types, composite types, array types, constraints,
type parameters, and labels. The primitive types identified are:
int/unit 8/16/32/64/128, and these types act as expected. Set types are
defined as specializations on existing types. An enum type, which is 
part of the set type category, allows specialization of a type with
a set of explicit values. Bitmask types are not explained in the paper.

Composite types are defined with records and variants. Records impose
order, no padding, no packing, etc. Nesting and anonymous composites are
allowed in the language. Array types are contiguous arrays of homogeneous
type. Arrays may be defined as having a lower and upper bound, where omission
of a lower bound implies a lower bound of 0.

Constraints provide an ability to tag declarations with predicate expression
asserting a match. These expressions include the type C operators along with
user defined predicates, sizeof, is, and forall. The function sizeof returns
the number of bytes of the given term. The predicate is determines type
membership for a term of a variant declaration. Forall is the universal
quantification operator that predicates over all elements of a range. For a
term to match against data, its constraints must resolve to true. Empty 
constraints are always true.

Type parameters are introduced as a mechanism to make type construction
parametric over terms. Labels are introduced as a mechanism to allow arbitrary 
offsets of terms. A label is used to decorate any declaration within a record to
define its beginning address relative to the beginning of the containing record.

Byte order of a declaration may be specified with a prefix. The default order is
taken from the order of the containing composite type, otherwise it is set to
Big Indean.

The DataScript compiler generates a Java library based on the specification
file. The library includes classes for encoding and decoding of data, as well
as a visitor for an instance of data.

\section{Critique}

This work contributes the language feature of the label, along with increased
readability over PacketTypes. Labels seem to be the primary contribution of this
work. Labels are a way to decorate an attribute of a record with a specific
starting offset relative to the beginning of the structure.

DataScript uses different terminology but provides largely the
same feature set that PacketTypes provides, with the exception
of labels. The paper claims to introduce `type parameters', however
on closer inspection this is just a form of type shape dependency
over previously occurring terms, which is provided in PacketTypes.

