\section{Background}
     
specification languages. Determining if the input specification is well-formed
then becomes a variation of type-checking. This phase allows a user of the
system to receive the benefit of sophisticated verification techniques, without
having understand them. A similar approach has been successful for defining
A protocol specification can be broken into four components: message format,
initial configuration, state machine, and system interface. In am currently in
the initial phase of my research where I am focusing on the message format
portion of protocol specifications. Packet Types [4] is one of the earliest
works where the authors show productivity benefits of having network types in an
overlay language. PADS [5] is a later work that showed formal treatment of 
network specific types such as Packet Types. I am currently working on a paper
that leverages these ideas that eliminate entire categories of common security
vulnerabilities for network protocol message formats. In particular there are
two common security vulnerabilities that occur in almost all network protocol
implementations: value constraint, and structural constraint violations. A
protocol specification might define a subset of the input range for any
particular value, leaving the remainder unspecified, value constraints are
violated when an implementation blindly handles received values without range
validation allowing for undefined behavior. A structural constraint violation
occurs when a received message describes its structure in a way that violates
the structural rules of the format specification. My first paper focuses on
defining a language for a large subset of IETF binary format based protocols
with the guarantee all well- formed specifications will not allow any type of
value or structural constraint violation. The US- CERT vulnerability database
shows these classes of vulnerabilities exist in current implementations of
simple, old, well understood protocols by extremely knowledgeable organizations [1].

\subsection{Network Protocol Problems }

\begin{tabular}{|c|c|c|c|c|}
   \hline
   Age & Protocol & Vendor & Bug Date & US CERT \# \\ \hline
   \hline
   1985 & Bootp & Apple & 2006 & 77628 \\ \hline
   1990 & 802.1q VLAN & Cisco & 2006 & 821420 \\ \hline
   1981 & ICMP & Cisco & 2007 & 341288 \\ \hline
   1985 & NTPD & GNU & 2009 & 853097 \\ \hline
   1998 & OSPFv2 & Quagga & 2012 & 551715 \\ \hline
\end{tabular}

\subsection{Prior Work}

Stuff \cite{packet_types}
