\section{Overview}

The current process of design, development and verification of network protocols
and network systems is extremely inefficient, error prone, and does not scale
well for large networks. This problem is further complicated by modern network
applications, which
involve complex interactions between many protocols. Long Term
Evolution (LTE) is the technology behind 4th Generation wireless broadband
networks. A single data flow in a LTE network requires the explicit interaction
of over 13 unique protocols communicating between a minimum of 7 unique network
element types. When the mobile device establishes a voice call the protocol
interaction grows even higher. This level of complexity is not unique to LTE,
but is a trend of modern networks, it can be observed in VoIP, IPTV, broadband
mobility, and content distribution network architectures. This increase in
complexity is detrimental to the comprehension, cost, and time of developing
these architectures. 

First, a protocol specification language will be developed to express protocol
design. It will be powerful enough to express the definition of a protocol
including: message formats, state machine, initial configuration, and system
interaction. Second, the specification language should support reasoning over a
protocols properties as well as over interactions with other protocols. Third,
rapid prototyping will be supported, allowing a set of specifications to be
compiled to a low-level language for testing. This framework would allow for
students, researchers, and industry developers to quickly define a protocol, or
network of protocols, reason about their properties and interactions, and
generate prototypes for network testing.

My research will leverage advanced programming language and verification
techniques for the specification, verification, and synthesis of network
protocols and network systems, without requiring users to understand complicated
logic systems or formal verification. In my research I will design a
light-weight expressive specification language that can be used to specify
network protocols and network systems. I will be able to guarantee that a
well-formed specification cannot exhibit entire categories of known problems.
This is achieved by building on theorem proving techniques and proof assistants
from the programming language community, to aid in the development of network
specification languages. Determining if the input specification is well-formed
then becomes a variation of type-checking. This phase allows a user of the
system to receive the benefit of sophisticated verification techniques, without
having understand them. A similar approach has been successful for defining
routing protocols in meta-routing [2] where any well-formed routing
specification will always converge. From a specification an implementation must
be synthesized, and its semantics must be equivalent with the specification.
Again in my research I will leverage techniques from the programming language
and verification community to show semantic equivalence of the synthesized 
system, such as the semantic equivalence verification techniques used by the
COMPCERT C compiler [3].

A protocol specification can be broken into four components: message format,
initial configuration, state machine, and system interface. In am currently in
the initial phase of my research where I am focusing on the message format
portion of protocol specifications. Packet Types [4] is one of the earliest
works where the authors show productivity benefits of having network types in an
overlay language. PADS [5] is a later work that showed formal treatment of
network specific types such as Packet Types. I am currently working on a paper
that leverages these ideas that eliminate entire categories of common security
vulnerabilities for network protocol message formats. In particular there are
two common security vulnerabilities that occur in almost all network protocol
implementations: value constraint, and structural constraint violations. A
protocol specification might define a subset of the input range for any
particular value, leaving the remainder unspecified, value constraints are
violated when an implementation blindly handles received values without range
validation allowing for undefined behavior. A structural constraint violation
occurs when a received message describes its structure in a way that violates
the structural rules of the format specification. My first paper focuses on
defining a language for a large subset of IETF binary format based protocols
with the guarantee all well- formed specifications will not allow any type of
value or structural constraint violation. The US- CERT vulnerability database
shows these classes of vulnerabilities exist in current implementations of
simple, old, well understood protocols by extremely knowledgeable organizations [1].
