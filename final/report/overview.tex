\section{Overview}

The current process of design, development and verification of network protocols
and network systems is extremely inefficient, error prone, and does not scale
well for large networks. This problem is further complicated by modern network
applications, which
involve complex interactions between many protocols. Long Term
Evolution (LTE) is the technology behind 4th Generation wireless broadband
networks. A single data flow in a LTE network requires the explicit interaction
of over 13 unique protocols communicating between a minimum of 7 unique network
element types. When the mobile device establishes a voice call the protocol
interaction grows even higher. This level of complexity is not unique to LTE,
but is a trend of modern networks, it can be observed in VoIP, IPTV, broadband
mobility, and content distribution network architectures. This increase in
complexity is detrimental to the comprehension, cost, and time of developing
these architectures.
This problem can be mitigated with a network protocol specification language.

Steve, is such a language, and was crafted to support protocol design,
verification, and testing. Initially, the language focuses on definition and use
of message formats. The language supports compile time reasoning over the
consistency of message format declarations as well as the consistent use of
user defined types. This work provides formal statics for Steve, as
well as an implementation of these rules.
This work leverages existing research in dependent types, dependent records, and
ad-hoc data description languages.

There is plenty of existing research on improving the quality of network
software through formal reasoning; however, most of this work requires 
familiarity with advanced formal verification techniques. Steve discharges the
user of this burden by lowering reasoning about the protocol into the host 
language and allowing for automated verification by the compiler. Similar
approaches have been successful for defining routing protocols 
(meta-routing~\cite{meta_routing}, NDLog~\cite{ndlog}) and compilers
(CompCert~\cite{compcert}).
