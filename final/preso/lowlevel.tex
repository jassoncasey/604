\frame{\frametitle{Base Environment \& Rules}
   \begin{flushleft}
   \begin{align*}
      \Gamma_0 &::= \{  \mbox{Nat}::\star, \mbox{Bool}::\star, 
                        \mbox{Ascii}::\star, \mbox{Uint x}::\star\} \\
      \Gamma_1 &::= \Gamma_0, \{ 0..\infty:\mbox{Nat}, True:\mbox{Bool}, 
                                 False:\mbox{Bool} \}
   \end{align*}
   \end{flushleft}

\begin{mathpar}
   \inferrule[Kind Basis] {
      \tau :: K \in \Gamma 
   } {
      \Gamma \vdash \tau :: K } \\
   \inferrule[Type Basis] {
      x : \tau \in \Gamma 
   } {
      \Gamma \vdash x : \tau } \\
\end{mathpar}
}

\frame{\frametitle{Abstraction \& Let Rules}
\begin{mathpar}
   \inferrule[Kind Abs]{
      \Sigma | \Gamma \vdash \tau_1 ::\star \\
      \Sigma | \Gamma,x:\tau_1 \vdash \tau_2 ::\star
   } {
      \Sigma | \Gamma \vdash \Pi x:\tau_1.\tau_2 ::\star 
   }\\
   \inferrule[Type Abs]{
      \Sigma | \Gamma \vdash \tau_1 ::\star \\
      \Delta | \Sigma | \Gamma,x:\tau_1 \vdash t ~:~\tau_2
   } {
      \Delta | \Sigma | \Gamma \vdash \lambda x:\tau_1.t ~:~ \Pi x:\tau_1.\tau_2
   }\\
   \inferrule[Type Let]{
      \Delta | \Sigma | \Gamma \vdash t_1 : \tau_1 \\
      \Sigma | \Gamma \vdash \tau_1 ::\star \\
      \Delta | \Sigma | \Gamma, \langle x:\tau_1 \rangle \vdash t_2 :\tau_2
   } {
      \Delta | \Sigma | \Gamma \vdash \mbox{let }x = t_1 \mbox{ in } t_2 : \tau_2 
   }\\
   \inferrule[Type TopLevel Let]{
      \Sigma | \Gamma \vdash \tau ::\star \\
      \Delta | \Sigma | \Gamma \vdash t : \tau \\
      \Delta | \Sigma | \Gamma, \langle f:\tau \rangle \vdash prgm
   } {
      \Delta | \Sigma | \Gamma \vdash \mbox{let }f = t:\tau, prgm 
   }
\end{mathpar}
}

\frame{\frametitle{Delta Functions}
   \scriptsize
   \begin{flushleft}
   \begin{align*}
      \mbox{nat\_plus}&: \Pi a:(\mbox{Nat}).\Pi b:(\mbox{Nat}).\mbox{Nat} \\
      \mbox{uint}&: \Pi a:(\mbox{Nat}).\Pi b:(\mbox{Nat}).\mbox{Uint}~a\\
      \mbox{uint\_plus}&: \Pi a:(\Pi x:\mbox{Nat}.\mbox{Uint}~x).
               \Pi b:(\Pi y:\mbox{Nat}.\mbox{Uint}~y).\mbox{Uint}~x>y?x+1:y+1\\
      \mbox{uint\_bits}&: \Pi a:(\Pi x:\mbox{Nat}.\mbox{Uint }x).\mbox{Nat}
   \end{align*}
   \end{flushleft}
}

\frame{\frametitle{Application Rules}
   \begin{mathpar}
   \inferrule[Term App]{
      \Delta | \Sigma | \Gamma | \theta \vdash t_1 : \Pi x:\tau_1.\tau_3 
      \rightsquigarrow \theta'\\
      \Delta | \Sigma | \Gamma | \theta' \vdash t_2:\tau_2 
      \rightsquigarrow \theta''\\
      bw \leftarrow BoundVar(\tau_1)~
      bv' \leftarrow Fresh(bv) \\
      \Delta | \Sigma | \Gamma | \theta'' \vdash [bv'/bv]Body(\tau_1) \equiv \tau_3 
      \rightsquigarrow \theta'''
   } {
      \Delta | \Sigma | \Gamma | \theta \vdash t_1~t_2:[t_2/x][bv'/bv]\tau 
      \rightsquigarrow \theta'''
   }
   \end{mathpar}
}

\frame{\frametitle{Predicate Introduction}
 \begin{itemize}
   \item $\Delta$ - set of known facts
   \item conditional terms introduce predicates
   \item evaluate sub-terms using new fact established by predicate
\end{itemize}

\begin{mathpar}
   \inferrule[Conditional]{
      \Delta | \Sigma | \Gamma | \theta \vdash t_1 : \mbox{Bool}
      \rightsquigarrow \theta' \\
      \Delta, \langle t_1 \rangle | \Sigma | \Gamma | \theta' \vdash
      t_2 : \tau \rightsquigarrow \theta'' \\
      \Delta, \langle t_1 \rangle | \Sigma | \Gamma | \theta'' \vdash
      t_3 : \tau \rightsquigarrow \theta''' \\
   } {
      \Delta | \Sigma | \Gamma | \theta \vdash \mbox{if } t_1
      \mbox{ else } t_2 \mbox{ then } t_3 : \tau
      \rightsquigarrow \theta'''
   }
\end{mathpar}

}

\frame{\frametitle{Type Equivalence Rules}
   \begin{mathpar}
   \inferrule[TAUT]{}{
      \Delta | \Sigma | \Gamma | \theta \vdash \tau \equiv \tau 
      \rightsquigarrow \theta
   } \\
   \inferrule[Pi]{
      \Delta | \Sigma | \Gamma | \theta \vdash \tau_1 \equiv \tau_3 
      \rightsquigarrow \theta'\\
      \Delta | \Sigma | \Gamma,\langle x:\tau_1 \rangle | \theta' \vdash 
      \tau_2 \equiv \tau_4 \rightsquigarrow \theta''
   }{
      \Delta | \Sigma | \Gamma | \theta \vdash \Pi x:\tau_1.\tau_2 \equiv
      \Pi x:\tau_3.\tau_4 \rightsquigarrow \theta''
   } \\
   \inferrule[Uint]{
      \Delta | \Sigma | \Gamma | \theta \vdash t_1 \equiv t_2
      \rightsquigarrow \theta'
   }{
      \Delta | \Sigma | \Gamma | \theta \vdash \mbox{Uint }t_1\equiv \mbox{Uint }t_2
      \rightsquigarrow \theta'
   } \\
   \inferrule[NatUint]{}{
      \Delta | \Sigma | \Gamma | \theta \vdash \mbox{Nat }\equiv \mbox{Uint }t
      \rightsquigarrow \theta
   } 
   \end{mathpar}
}

\frame{\frametitle{Term Equivalence Rules}
   \begin{mathpar}
   \inferrule[TAUT]{}{
      \Delta | \Sigma | \Gamma | \theta \vdash x \equiv x
      \rightsquigarrow \theta
   } \\
   \inferrule[TermConstant]{
      x \notin \Delta \\
      NormalForm(t) \\
      x \notin dom(\theta)
   }{
      \Delta | \Sigma | \Gamma | \theta \vdash x \equiv t
      \rightsquigarrow \theta,\langle x:t\rangle
   } \\
   \inferrule[CondElim]{
      \Delta \models t_2 \\
      \Delta | \Sigma | \Gamma | \theta \vdash t_1 \equiv t_3
      \rightsquigarrow \theta'
   } {
      \Delta | \Sigma | \Gamma | \theta \vdash t_1 \equiv t_2 ? t_3 : t_4
      \rightsquigarrow \theta'
   } \\
   \inferrule[TermConstant]{
      \Delta | \Sigma | \Gamma | \theta \vdash AppOrder(t_1) \equiv AppOrder(t_2) 
      \rightsquigarrow \theta'
   }{
      \Delta | \Sigma | \Gamma | \theta \vdash t_1 \equiv t_2
      \rightsquigarrow \theta'
   }
   \end{mathpar}
}
