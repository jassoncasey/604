\documentclass{beamer}
\usepackage{bussproofs}
\usetheme{default}

\title{Summarizing: OutsideIn(X): Modular Type Inference with Local Assumptions.}
\subtitle{D. Vytiniotis, S. P. Jones, T. Schrijvers, M. Sulzmann}
\author{Jasson Casey \& Michael Lopez}
\date{April 26, 2012}

\begin{document}
\begin{frame}
\titlepage
\end{frame}

\begin{frame}{Problems with HM System}
\end{frame}

\begin{frame}{Constraints}
\begin{itemize}
\item Constraints are used to solve type inference
\item Sample constraint: X = $\{ \tau_1 \sim \tau_2\}$
\end{itemize}

\begin{prooftree}
\def \fCenter{\ \vdash\ }
\RightLabel{\textbf{App}}
\AxiomC{$\Gamma \fCenter\ t_1 : \alpha_1 \rightarrow \alpha_2$}
\AxiomC{$\Gamma \fCenter\ t_2 : \alpha_3$}
\AxiomC{$X=\{\alpha_1 \sim \alpha_3\}$}
\TrinaryInfC{$ \Gamma \fCenter\ t_1$ $t_2 : \alpha_2$}
\end{prooftree}

\begin{itemize}
\item Type systems are parametric over constraints
\item Examples: HM(X) and InsideOut(X)
\end{itemize}

\end{frame}

\begin{frame}{Constraint based Type System}
\begin{itemize}
\item Typeing Relation: Q, $\Gamma \vdash e : \tau$
\item Q = constraint set
\item $\Gamma=$ type environment
\item $\tau=$ resulting type
\item Soundness of system depends on consistency of constraints
\end{itemize}

\end{frame}

\begin{frame}{Constraint Logic}
\begin{itemize}
\item A language of types and terms
\item term: $f=\tau \sim \tau | f \wedge f$
\item Type equality, $\sim:$ is commutative and associative
\item Conjunction, $\wedge:$ is commutative and associative
\item Deductions: $\wedge_{introduction},$ $\wedge_{elimination}$
\end{itemize}
\end{frame}

\begin{frame}{Constrained Types}
\begin{itemize}
\item $\mathcal{Q}$: top level constraint set
\item Q: the set of type constraints (flat level)
\item Constrained type: $\forall \bar{a}. Q \Rightarrow \tau$
\item $\bar{a}:$ a tuple of type variables
\end{itemize}
\end{frame}

\begin{frame}{Type Checking Type Classes}
\begin{prooftree}
\def \fCenter{\ \vdash\ }
\RightLabel{VarCon}
\AxiomC{}
\AxiomC{}
\BinaryInfC{$Q,\Gamma \fCenter\ == :$}
\AxiomC{$Q,\Gamma \fCenter\ 1: Int$}
\RightLabel{App}
\BinaryInfC{$Q,\Gamma \fCenter\ == 1: $}
\AxiomC{$Q,\Gamma \fCenter\ 2: Int$}
\RightLabel{App}
\BinaryInfC{$Q,\Gamma \fCenter\ == 1$ $2 : $}
\end{prooftree}
\end{frame}

\end{document}
