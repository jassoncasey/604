\documentclass{beamer}
\usepackage{bussproofs}
\usepackage{listings}
\usetheme{default}

\title{Summarizing: OutsideIn(X): Modular Type Inference with Local Assumptions.}
\subtitle{D. Vytiniotis, S. P. Jones, T. Schrijvers, M. Sulzmann}
\author{Jasson Casey \& Michael Lopez}
\date{April 26, 2012}

\begin{document}
\begin{frame}
\titlepage
\end{frame}

\begin{frame}
\frametitle{Modular Type Inference}
\begin{itemize}
\item Infer type of function without regard to call sites
\item Useful for any production language
\item Separate compilation, large files
\end{itemize}
\end{frame}

\begin{frame}
\frametitle{Local Assumptions}
\begin{itemize}
\item Let introduces new types and constraints
\item Pattern matching GADTs introduces new types and constraints
\item Functions using GADTs can lack principle types
\item Axioms schemes introduce infinite recursive types, having no principle types
\end{itemize}
\end{frame}

\begin{frame}{New Problems for Hindley Milner}
HM(X) is not powerful enough to support the features present in modern day languages. Some of the features include
\begin{itemize}
\item GADTs
\item Multi-parameter type classes
\item Data constructors with existential types
\end{itemize}
Split into 2 slides
Reason: HM(X) is unable to deal with local constraints and top-level axioms. We will see an example of this soon.

Languages like GADTs are {\emph good} features. They make the language more expressive at the cost of some complexity.
\end{frame}

%%%%%%%%
\begin{frame}{New Problems for Hindley Milner}
  Reason: Type constraints are no longer restricted to structural equality ($\sim$). There are now
  \begin{itemize}
    \item Local constraints
    \item Top-level axioms
  \end{itemize}
  Goal: Create a type system that supports these constraints without adding too much complexity.
\end{frame}

%%%%%%%%%
\begin{frame}[fragile]
\frametitle{Modular Type Inference}
One of the big problems that GADTs present is the loss of most principled types. Consider the following.
\begin{lstlisting}
data T a where
  T1 :: Int -> T Bool
  T2 :: T a

test (T1 n) _ = n >0
test T2     r = r
\end{lstlisting}
What is the type prinicple type of function test?
\end{frame}

%%%%%%%%%
\begin{frame}[fragile]
\frametitle{Modular Type Inference}
One of the big problems that GADTs present is the loss of principled types. Consider the following.
\begin{lstlisting}
data T a where
  T1 :: Int -> T Bool
  T2 :: T a

test (T1 n) _ = n > 0
test T2     r = r
\end{lstlisting}
The function test can be either
\begin{itemize}
\item $\forall \mbox{T} \alpha.\alpha \rightarrow \mbox{Bool} \rightarrow \mbox{Bool}$
\item $\forall \mbox{T} \alpha.\alpha \rightarrow \alpha \rightarrow \alpha$
\end{itemize}
Since neither type is an instantiation of the other, the type of test cannot be inferred. Should we require annotation on all functions with local constraints?
\end{frame}

%%%%%%%%%%
\begin{frame}[fragile]
\frametitle{Modular Type Inference}
One of the big problems that GADTs present is the loss of most principled types. Consider the following.
\begin{lstlisting}
data T a where
  T1 :: Int -> T Bool
  T2 :: T a

test2 (T1 n) _ = n > 0
test2 T2     r = not r
\end{lstlisting}
Still using GADTs, but the function not forces
\begin{itemize}
\item $\forall \mbox{T} \alpha.\alpha \rightarrow \mbox{Bool} \rightarrow \mbox{Bool}$
\end{itemize}
to be infered.
\end{frame}


%%%%%%%%
\begin{frame}[fragile]
\frametitle{Top-level Axioms}
Consider the following
\begin{lstlisting}
class Foo a b where foo :: a -> b -> Int
instance Foo Int b
instance Foo a b => Foo [a] b

g y = let h :: forall c. c -> Int
          h x = foo y x
  in h True
\end{lstlisting}
What is the type of g?
\end{frame}

\begin{frame}[fragile]
\frametitle{Top-level Axioms}
Consider the following
\begin{lstlisting}
class Foo a b where foo :: a -> b -> Int
instance Foo Int b
instance Foo a b => Foo [a] b

g y = let h :: forall c. c -> Int
          h x = foo y x
  in h True
\end{lstlisting}
We can deduce that g may have the type $[\mbox{Int}] \rightarrow \mbox{Int}$.
\end{frame}

\begin{frame}[fragile]
\frametitle{Top-level Axioms}
Consider the following
\begin{lstlisting}
class Foo a b where foo :: a -> b -> Int
instance Foo Int b
instance Foo a b => Foo [a] b

g y = let h :: forall c. c -> Int
          h x = foo y x
  in h True
\end{lstlisting}
We can deduce that g may have the type $[\mbox{Int}] \rightarrow \mbox{Int}$.

Applying the axiom scheme given by Foo again, $[[\mbox{Int}]] \rightarrow \mbox{Int}$ is also a possible type for g.
\end{frame}

\begin{frame}{Constraints}
\begin{itemize}
\item Constraints are used to solve type inference
\item Sample constraint: X = $\{ \tau_1 \sim \tau_2\}$
\end{itemize}

\begin{prooftree}
\def \fCenter{\ \vdash\ }
\RightLabel{\textbf{App}}
\AxiomC{$\Gamma \fCenter\ t_1 : \alpha_1 \rightarrow \alpha_2$}
\AxiomC{$\Gamma \fCenter\ t_2 : \alpha_3$}
\AxiomC{$X=\{\alpha_1 \sim \alpha_3\}$}
\TrinaryInfC{$ \Gamma \fCenter\ t_1$ $t_2 : \alpha_2$}
\end{prooftree}

\begin{itemize}
\item Type systems are parametric over constraints
\item Examples: HM(X) and InsideOut(X)
\end{itemize}

\end{frame}

\begin{frame}{Constraint based Type System}
\begin{itemize}
\item Typeing Relation: Q, $\Gamma \vdash e : \tau$ or $\Gamma \vdash e : \tau \rightsquigarrow C$
\item Q = constraint set
\item C = generated constraint set
\item $\Gamma=$ type environment
\item $\tau=$ resulting type
\item Soundness of system depends on consistency of constraints
\end{itemize}
\end{frame}

\begin{frame}{Constrained Types}
\begin{itemize}
\item $\mathcal{Q}$: top level constraint set
\item $\mathcal{Q}=\{Eq$ $Int, Eq$ $Char\}$
\end{itemize}
\begin{itemize}
\item Q: the set of type constraints (flat level)
\item Q=$\{\tau_1 \sim Int\}$
\end{itemize}
\begin{itemize}
\item Constrained type: $\forall \bar{a}. Q \Rightarrow \tau$
\item $\bar{a}:$ a tuple of type variables
\item == : $\forall \alpha. Eq$ $\alpha \Rightarrow \alpha \rightarrow \alpha \rightarrow Bool$
\end{itemize}
\end{frame}

\begin{frame}{Constraint Logic}
\begin{itemize}
\item A language of types and terms
\item term: $f=\tau \sim \tau | f \wedge f$
\item Type equality, $\sim:$ is commutative and associative
\item Conjunction, $\wedge:$ is commutative and associative
\item Deductions: $\wedge_{introduction},$ $\wedge_{elimination}$
\end{itemize}
\end{frame}

\begin{frame}{Type Inference Type Classes}
\begin{itemize}
\item Generate constraints in pass 1
\item $\Gamma \vdash t:\tau \rightsquigarrow Constraints$
\item $\Gamma \vdash ==1$ $2: \tau \rightsquigarrow Constraint$
\end{itemize}
\begin{prooftree}
\def \fCenter{\ \vdash\ }
\scriptsize 
\RightLabel{VarCon}
\AxiomC{(==:$\forall \alpha .Eq \alpha \Rightarrow \alpha \rightarrow \alpha \rightarrow Bool) \in dom(\Gamma)$}
\UnaryInfC{$\Gamma \fCenter\ == : \alpha_1 \rightarrow \alpha_1 \rightarrow Bool
            \rightsquigarrow Eq \alpha_1$}
\RightLabel{App}
\UnaryInfC{$\Gamma \fCenter\ == 1: \alpha_2 \rightsquigarrow Eq \alpha_1 \wedge
            \alpha_1 \rightarrow \alpha_1 \rightarrow Bool \sim (Int \rightarrow \alpha_2)$}
\RightLabel{App}
\UnaryInfC{$\Gamma \fCenter\ == 1$ $2 : \alpha_3 \rightsquigarrow Eq \alpha_1 \wedge
            \alpha_1 \rightarrow \alpha_1 \rightarrow Bool \sim (Int \rightarrow \alpha_2) \wedge
            \alpha_2 \sim (Int \rightarrow \alpha_3)$}
\end{prooftree}
\end{frame}

\begin{frame}{Type Inference with Type Classes}
\begin{itemize}
\item Solve constraints in pass 2
\end{itemize}
$\mathcal{Q}=\{Eq$ $Int\}$\\
$Q_{given}=\emptyset$\\
$Q_{wanted}=Eq \alpha_1, \alpha_1 \rightarrow \alpha_1 \rightarrow Bool \sim 
            (Int \rightarrow \alpha_2), \alpha_2 \sim (Int \rightarrow \alpha_3)$\\
$Q_{residual}=\emptyset$\\
$\theta=\{\alpha_1=Int,\alpha_2=Int \rightarrow \alpha_3, \alpha_3=Bool\}$\\
Validate = $[\alpha_1 \mapsto Int]\{Eq \alpha_1\} \in \mathcal{Q}$
\end{frame}

\begin{frame}{Locally Constrained Types}
\begin{itemize}
\item Where do local assumptions come from?
\end{itemize}

\begin{itemize}
\item Let bindings
\item Pattern match of data constructors
\item Local type variables that may have non-local dependencies
\end{itemize}
\end{frame}

\begin{frame}[fragile]{Local Constraints from Let Bindings}
\begin{verbatim}
fr x y = let gr z = not x in ...
\end{verbatim}

\begin{prooftree}
\def \fCenter{\ \vdash\ }
\AxiomC{$not \in dom(\Gamma)$}
\UnaryInfC{$\Gamma \fCenter\ not : Bool \rightarrow Bool$}
\AxiomC{$\{x:\alpha_1\} \fCenter\ x : \alpha_1$}
\BinaryInfC{$\Gamma,\{z:\alpha_3\} \fCenter\ not$ $x : Bool$}
\UnaryInfC{$\{x:\alpha_1,y:\alpha_2\} \fCenter\ \lambda$ $z. not$ $x: \alpha_3 \rightarrow Bool$}
\end{prooftree}

\begin{itemize}
\item $\Gamma \vdash\ gr: \forall \alpha.\alpha \rightarrow Bool$
\item But what about the constraint: $\{Bool \sim \alpha_1\}$
\end{itemize}
\end{frame}

\begin{frame}{Why Modular Inference Fails}
\begin{itemize}
\item $\Gamma \vdash\ gr: \forall \alpha.\alpha \rightarrow Bool$
\item This type fails to capture any information about $\alpha_1$
\item Any application will type check
\item However, $\alpha_1$ has required type and must be part application type checking
\item A better type: $\Gamma \vdash\ gr: \forall \alpha. Bool \sim \alpha \Rightarrow \alpha \rightarrow Bool$
\end{itemize}
\end{frame}

\begin{frame}[fragile]
\frametitle{Source of Problems}
\begin{itemize}
\item GADTs - Type constructors with refined type
\begin{itemize}
\item \begin{lstlisting}
T1 :: (a ~ Bool) => Int -> T a
\end{lstlisting}
\end{itemize}
\item Type-classes - Use of type class functions
\begin{itemize}
\item $\forall \alpha . \mbox{Eq} \alpha \Rightarrow \alpha \rightarrow \alpha \rightarrow \mbox{Bool}$
\end{itemize}
\item Axioms - infinitely many derivable 
\begin{itemize}
\item \begin{lstlisting} instance Foo Int b
instance Foo a b => Foo [a] b
\end{lstlisting}
\end{itemize}
\end{itemize}
\end{frame}

\begin{frame}{Solution Objectives}
\begin{itemize}
\item HM type inference cannot handle constraints
\item Constraints must be part of type inference and checking
\item Quantification is desirable but can cause problems
\end{itemize}
\end{frame}

\begin{frame}{OutsideIn - in 3 phases}
\begin{enumerate}
   \item Infer types and gather constraints
   \item Unify gathered constraints
   \item Type check against learned types
\end{enumerate}
\end{frame}

\begin{frame}{Type Inference \& Unification}
\begin{itemize}
\item $\mathcal{Q};\Gamma \triangleright prog$
\item $\Gamma \triangleright t : \tau \rightsquigarrow C$
\item $\mathcal{Q}$ - global knowledge
\end{itemize}

\begin{prooftree}
\scriptsize
\RightLabel{$Bind_{top level}$}
\AxiomC{$\Gamma \triangleright t : \tau \rightsquigarrow C$}
\AxiomC{...}
\BinaryInfC{$\mathcal{Q};\Gamma \triangleright f=t,prog$}
\end{prooftree}

\end{frame}

\begin{frame}{Inference \& Unification}
\begin{itemize}
\item $\mathcal{Q};Q_{given};\bar{\alpha_{tch}} \triangleright^{unify} C_{wanted} \rightsquigarrow Q_{residual};\theta$
\item $\mathcal{Q}$ - global knowledge
\item $Q$ - local knowledge
\end{itemize}
\begin{prooftree}
\scriptsize
\RightLabel{$Bind_{top level}$}
\AxiomC{...}
\AxiomC{$\mathcal{Q};\epsilon;fuv(\tau,C)\triangleright^{unify} C \rightsquigarrow Q;\theta$}
\BinaryInfC{$\mathcal{Q};\Gamma \triangleright f=t,prog$}
\end{prooftree}

\begin{prooftree}
\scriptsize
\RightLabel{$Bind_{top level}$}
\AxiomC{$\bar{a}$ fresh}
\AxiomC{$\bar{\alpha} =fuv(\theta \tau,Q)$}
\AxiomC{$\mathcal{Q};\Gamma,(f:\forall\bar{a}.[\bar{\alpha \mapsto a}](Q \Rightarrow \theta \tau)) \triangleright prog$}
\TrinaryInfC{$\mathcal{Q};\Gamma \triangleright f=t,prog$}
\end{prooftree}

\end{frame}

\begin{frame}{Inference \& Unification}
\begin{itemize}
\item let expects type annotation
\item at least one constraint or quantified type
\end{itemize}

\begin{prooftree}
\scriptsize
\RightLabel{$GLetA_{flat level}$}
\AxiomC{$\sigma_1 =\forall \bar{a}.Q_1 \Rightarrow \tau_1$}
\AxiomC{$Q_1 \neq \epsilon$ or $\bar{a} \neq \epsilon$}
\AxiomC{$\Gamma \triangleright e_1 : \tau \rightsquigarrow C$}
\TrinaryInfC{$\Gamma \triangleright$ let x = $e_1 :: \sigma_1$ in $e_2 : \tau_2 \rightsquigarrow C_1 \wedge C_2$}
\end{prooftree}

\begin{itemize}
\item all local type variables are quantified
\item Local constraints are predicated on type constraints
\end{itemize}

\begin{prooftree}
\scriptsize
\AxiomC{$\bar{\beta} = fuv(\tau,C)-fuv(\Gamma)$}
\AxiomC{$C_1=\exists \bar{\beta}.(Q_1 \supset C \wedge \tau \sim \tau_1)$}
\AxiomC{$\Gamma,(x:\sigma_1) \triangleright e_2 : \tau_2 \rightsquigarrow C_2$}
\TrinaryInfC{$\Gamma \triangleright$ let x = $e_1 :: \sigma_1$ in $e_2 : \tau_2 \rightsquigarrow C_1 \wedge C_2$}
\end{prooftree}
\end{frame}

\begin{frame}
\frametitle{Soundness and Completeness}
\begin{itemize}
\item Soundness - If the type of an expression can be inferred, then it can be type checked.
\item Completeness - If the type of an expression can be type checked, then it can be inferred.
\end{itemize}
\end{frame}

\begin{frame}{Sound \& Complete}
\begin{itemize}
\item Shorthand
   \begin{itemize}
   \item $\Gamma \vdash t : \tau$ - type checking
   \item $\Gamma \vdash t \uparrow \tau$ - type inference
   \end{itemize}
\item Soundness
   \begin{itemize}
   \item if $\Gamma \vdash t \uparrow \tau$ then $\Gamma \vdash t : \tau$
   \end{itemize}
\item Completeness
   \begin{itemize}
   \item if $\Gamma \vdash t : \tau$ then $\Gamma \vdash t \uparrow \tau$
   \end{itemize}
\end{itemize}
\end{frame}

\begin{frame}
\frametitle{Soundness \& Completeness of OutsideIn}
\begin{itemize}
   \item OutsideIn is sound
      \begin{itemize}
         \item for all possible programs where inference produces a type, that same program will type check
      \end{itemize}
   \item OutsideIn is incomplete
      \begin{itemize}
         \item inference will fail to produce a type for some valid programs
      \end{itemize}
\end{itemize}
\end{frame}

\begin{frame}{Lost bits}
Many ideas and justification details are in the paper ...
\begin{itemize}
\item Complete rules of the OutsideIn system
\item Proofs of soundness for type checking or inference
\item Logic solver, its soundness, and decidability of termination
\end{itemize}
\end{frame}

\end{document}
