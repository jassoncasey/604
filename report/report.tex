\documentclass{beamer}
\usepackage{bussproofs}
\usepackage{listings}
\usetheme{default}

\title{Summarizing: OutsideIn(X): Modular Type Inference with Local Assumptions.}
\subtitle{D. Vytiniotis, S. P. Jones, T. Schrijvers, M. Sulzmann}
\author{Jasson Casey \& Michael Lopez}
\date{April 26, 2012}

\begin{document}
\begin{frame}
\titlepage
\end{frame}

\begin{frame}
\frametitle{Modular Type Inference}
\begin{itemize}
\item Infer type of function without regard to call sites
\item Useful for any production language
\item Separate compilation, large files
\end{itemize}
\end{frame}

\begin{frame}
\frametitle{Local Assumptions}
\begin{itemize}
\item Let introduces new types and constraints
\item Pattern matching GADTs introduces new types and constraints
\item Functions using GADTs can lack principle types
\item Axioms schemes introduce infinite recursive types, having no principle types
\end{itemize}
\end{frame}

\begin{frame}{New Problems for Hindley Milner}
HM(X) is no longer powerful enough to support the features present in modern day languages. Some of the features include
\begin{itemize}
\item GADTs
\item Multi-parameter type classes
\item Data constructors with existential types
\end{itemize}
Reason: HM(X) is unable to deal with local constraints and top-level axioms. We will see an example of this soon.

Languages like GADTs are {\emph good} features. They make the language more expressive at the cost of some complexity.
\end{frame}

\begin{frame}[fragile]
\frametitle{Modular Type Inference}
One of the big problems that GADTs present is the loss of most principled types. Consider the following.
\begin{lstlisting}
data T a where
  T1 :: Int -> T Bool
  T2 :: T a

test (T1 n) _ = n >0
test T2     r = r
\end{lstlisting}
What is the type prinicple type of function test?
\end{frame}

\begin{frame}[fragile]
\frametitle{Modular Type Inference}
One of the big problems that GADTs present is the loss of most principled types. Consider the following.
\begin{lstlisting}
data T a where
  T1 :: Int -> T Bool
  T2 :: T a

test (T1 n) _ = n > 0
test T2     r = r
\end{lstlisting}
The function test can be either
\begin{itemize}
\item $\forall \alpha.T\alpha \rightarrow \mbox{Bool} \rightarrow \mbox{Bool}$
\item $\forall \alpha.T\alpha \rightarrow \alpha \rightarrow \alpha$
\end{itemize}
Since neither type is an instantiation of the other, the type of test cannot be inferred.
\end{frame}

\begin{frame}{Top-level Axioms}

\end{frame}

\begin{frame}{Constraints}
\begin{itemize}
\item Constraints are used to solve type inference
\item Sample constraint: X = $\{ \tau_1 \sim \tau_2\}$
\end{itemize}

\begin{prooftree}
\def \fCenter{\ \vdash\ }
\RightLabel{\textbf{App}}
\AxiomC{$\Gamma \fCenter\ t_1 : \alpha_1 \rightarrow \alpha_2$}
\AxiomC{$\Gamma \fCenter\ t_2 : \alpha_3$}
\AxiomC{$X=\{\alpha_1 \sim \alpha_3\}$}
\TrinaryInfC{$ \Gamma \fCenter\ t_1$ $t_2 : \alpha_2$}
\end{prooftree}

\begin{itemize}
\item Type systems are parametric over constraints
\item Examples: HM(X) and InsideOut(X)
\end{itemize}

\end{frame}

\begin{frame}{Constraint based Type System}
\begin{itemize}
\item Typeing Relation: Q, $\Gamma \vdash e : \tau$ or $\Gamma \vdash e : \tau \rightsquigarrow C$
\item Q = constraint set
\item C = generated constraint set
\item $\Gamma=$ type environment
\item $\tau=$ resulting type
\item Soundness of system depends on consistency of constraints
\end{itemize}
\end{frame}

\begin{frame}{Constrained Types}
\begin{itemize}
\item $\mathcal{Q}$: top level constraint set
\item $\mathcal{Q}=\{Eq$ $Int, Eq$ $Char\}$
\end{itemize}
\begin{itemize}
\item Q: the set of type constraints (flat level)
\item Q=$\{\tau_1 \sim Int\}$
\end{itemize}
\begin{itemize}
\item Constrained type: $\forall \bar{a}. Q \Rightarrow \tau$
\item $\bar{a}:$ a tuple of type variables
\item == : $\forall \alpha. Eq$ $\alpha \Rightarrow \alpha \rightarrow \alpha \rightarrow Bool$
\end{itemize}
\end{frame}

\begin{frame}{Locally Constrained Types}
\begin{itemize}
\item Where do local constaints come from?
\item $\forall \bar{a}. Q \Rightarrow \tau$
\end{itemize}

\begin{itemize}
\item Let bindings
\item Pattern match of data constructors
\end{itemize}
\end{frame}

\begin{frame}{Constraint Logic}
\begin{itemize}
\item A language of types and terms
\item term: $f=\tau \sim \tau | f \wedge f$
\item Type equality, $\sim:$ is commutative and associative
\item Conjunction, $\wedge:$ is commutative and associative
\item Deductions: $\wedge_{introduction},$ $\wedge_{elimination}$
\end{itemize}
\end{frame}

\begin{frame}{Type Inference Type Classes}
\begin{itemize}
\item Generate constraints in pass 1
\item $\Gamma \vdash t:\tau \rightsquigarrow Constraints$
\item $\Gamma \vdash ==1$ $2: \tau \rightsquigarrow Constrained$
\end{itemize}
\begin{prooftree}
\def \fCenter{\ \vdash\ }
\scriptsize 1
\RightLabel{VarCon}
\AxiomC{(==:$\forall \alpha .Eq \alpha \Rightarrow \alpha \rightarrow \alpha \rightarrow Bool) \in dom(\Gamma)$}
\UnaryInfC{$\Gamma \fCenter\ == : \alpha_1 \rightarrow \alpha_1 \rightarrow Bool
            \rightsquigarrow Eq \alpha_1$}
\RightLabel{App}
\UnaryInfC{$\Gamma \fCenter\ == 1: \alpha_2 \rightsquigarrow Eq \alpha_1,
            \alpha_1 \rightarrow \alpha_1 \rightarrow Bool \sim (Int \rightarrow \alpha_2)$}
\RightLabel{App}
\UnaryInfC{$\Gamma \fCenter\ == 1$ $2 : \alpha_3 \rightsquigarrow Eq \alpha_1, 
            \alpha_1 \rightarrow \alpha_1 \rightarrow Bool \sim (Int \rightarrow \alpha_2), 
            \alpha_2 \sim (Int \rightarrow \alpha_3)$}
\end{prooftree}
\end{frame}

\begin{frame}{Type Inference with Type Classes}
\begin{itemize}
\item Solve constraints in pass 2
\end{itemize}
$\mathcal{Q}=\{Eq$ $Int\}$\\
$Q_{given}=\emptyset$\\
$Q_{wanted}=Eq \alpha_1, \alpha_1 \rightarrow \alpha_1 \rightarrow Bool \sim 
            (Int \rightarrow \alpha_2), \alpha_2 \sim (Int \rightarrow \alpha_3)$\\
$Q_{residual}=\emptyset$\\
$\theta=\{\alpha_1=Int,\alpha_2=Int \rightarrow \alpha_3, \alpha_3=Bool\}$\\
\end{frame}

\end{document}
